%-------------------------------------------------------------------------------
%	SECTION TITLE
%-------------------------------------------------------------------------------
\cvsection{Education}


%-------------------------------------------------------------------------------
%	CONTENT
%-------------------------------------------------------------------------------
\begin{cventries}

  % ---------------------------------------------------------
  \cventry
  {PhD, Biological and Medical Informatics} % Degree
  {University of California, San Francisco (UCSF)} % Institution
  {San Francisco, CA} % Location
  {September 2011 --- Exp. August 2017} % Date(s)
  {
    \begin{cvitems} % Description(s) bullet points
    \item Led a project within the lab to benchmark and assess computational protocols for protein structure prediction/design
    \item Created Rosetta design pipeline to
      engineer new protein-ligand interactions for biosensor applications
    \item Collaborated externally to alter protein ligand binding preference to allow enzyme function with novel substrates
    \item Created a novel conformational sampling Rosetta method to predict protein interface $\Delta\Delta G$ values by sampling backbone structure space
    \end{cvitems}
  }
  % ---------------------------------------------------------

  % ---------------------------------------------------------
  \cventry
  {Bachelor of Arts, Molecular and Cell Biology} % Degree
  {University of California, Berkeley} % Institution
  {Berkeley, CA} % Location
  {August 2006 --- May 2010} % Date(s)
  {
    % \vspace{-\baselineskip}
    \begin{cvitems} % Description(s) bullet points
    % \item Coursework: Biochemistry, Biology, Biophysical Chemistry, Calculus, Chemistry, Computer Science, Discrete Math and Probability, Organic Chemistry, Physics, Statistics
    \item Research experience at the Museum of Vertebrate Zoology Genetics Lab
    \item Sequenced 14 microsatellite loci in 200+ Ensatina salamanders in order to determine dispersal rate of hybrids and population structure of triple hybrid zone in the Sierras
    \item Analyzed genotypes and used computational models to estimate migration rate
    \end{cvitems}
  }
  % ---------------------------------------------------------

\end{cventries}
